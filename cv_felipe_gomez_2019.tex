%%%%%%%%%%%%%%%%%%%%%%%%%%%%%%%%%%%%%%%%%
% "ModernCV" CV and Cover Letter
% LaTeX Template
% Version 1.2 (25/3/16)
%
% This template has been downloaded from:
% http://www.LaTeXTemplates.com
%
% Original author:
% Xavier Danaux (xdanaux@gmail.com) with modifications by:
% Vel (vel@latextemplates.com)
%
% License:
% CC BY-NC-SA 3.0 (http://creativecommons.org/licenses/by-nc-sa/3.0/)
%
% Important note:
% This template requires the moderncv.cls and .sty files to be in the same 
% directory as this .tex file. These files provide the resume style and themes 
% used for structuring the document.
%
%%%%%%%%%%%%%%%%%%%%%%%%%%%%%%%%%%%%%%%%%

%----------------------------------------------------------------------------------------
%	PACKAGES AND OTHER DOCUMENT CONFIGURATIONS
%----------------------------------------------------------------------------------------

\documentclass[12pt,letterpaper,sans]{moderncv} 
% Font sizes: 10, 11, or 12; paper sizes: a4paper, letterpaper, a5paper, legalpaper, 
% executivepaper or landscape; font families: sans or roman

\usepackage[utf8]{inputenc}

\moderncvstyle{casual} 
% CV theme - options include: 'casual' (default), 'classic', 
% 'oldstyle' and 'banking'
\moderncvcolor{blue} 
% CV color - options include: 'blue' (default), 'orange', 'green', 'red', 'purple', 
% 'grey' and 'black'

\usepackage{lipsum} % Used for inserting dummy 'Lorem ipsum' text into the template

\usepackage[scale=0.75]{geometry} % Reduce document margins
%\setlength{\hintscolumnwidth}{3cm} % Uncomment to change the width of the dates column
%\setlength{\makecvtitlenamewidth}{10cm} % For the 'classic' style, uncomment to adjust 
% the width of the space allocated to your name

\setlength\parindent{0pt}
\newcommand{\forceindent}{\leavevmode{\parindent=1em\indent}} 
% Filis Var: To indent a single paragraph


%----------------------------------------------------------------------------------------
%	NAME AND CONTACT INFORMATION SECTION
%----------------------------------------------------------------------------------------

\firstname{Felipe L.} % Your first name
\familyname{G\'omez-Cort\'es} % Your last name

% All information in this block is optional, comment out any lines you don't need
\title{Curriculum Vitae}
\address{Calle 23C 70-50 Int 21 Apt 403}{Bogot\'a, Colombia.}
\mobile{(+57) 316 845 5161}
%\phone{(+571) 759 5221}
%\fax{(000) 111 1113}
\email{flgomezc87@gmail.com}
\homepage{github.com/flgomezc}{https://github.com/flgomezc} 
%\extrainfo{additional information}
%\photo[70pt][0.4pt]{pictures/picture} % The first bracket is the picture height, 
% the second is the thickness of the frame around the picture (0pt for no frame)
%\quote{"A witty and playful quotation" - John Smith}

%----------------------------------------------------------------------------------------

\begin{document}

%----------------------------------------------------------------------------------------
%	CURRICULUM VITAE
%----------------------------------------------------------------------------------------

\makecvtitle % Print the CV title

%----------------------------------------------------------------------------------------
%	PROFILE SECTION
%----------------------------------------------------------------------------------------

\section{Profile}

I am Physicist. My research area is Computational Astrophysics. I have developed different 
strong skills in order to solve physical problems; Mathematical Modeling, Data Analysis, 
Python and C++ programming under UNIX-like operating systems. 
\par
\forceindent I have taught at university level as as Graduate Student 
Assistant and Adjunct Instructor.

As graduate assistant I have taught at Universidad de Los Andes; a basic Python course for
science and engineering, the complementary session of the Computational Methods course (for
physics and geo-sciences students), the complementary session of Physics II (electricity,
magnetism and thermodynamics) and laboratory sessions for Physics I (mechanics) and Physics II. 

\forceindent As entrepreneur I have learned the
Lean Startup methodology, also I have got software developing skills as in back-end as 
well in front-end. I have worked in a couple projects involving Apache, Java (Android),
MySQL, PHP, NodeJS and JavaScript (Facebook Messenger Chatbots).
\par

%----------------------------------------------------------------------------------------
%	WORK EXPERIENCE SECTION
%----------------------------------------------------------------------------------------

\section{Work Experience}

\cventry{2019 Jan. - Up to date}
        {Python for Science professor}
        {\textsc{Universidad de Los Andes}}
        {Bogotá, Colombia}{}
        {The course ``Herramientas Computacionales'' (Computing Tools) is an introductory course
          to python programming in Science. The students learn the basics of UNIX like operative
          systems, Python 3.7 and its interactive version IPython via Jupyter Notebooks, how to
          use Matplotlib, basic statistics, optimization and MoteCarlo Methods. This course was
          created for physics undergrad students, but also natural science and engineering
          undergrad students are attending.}

\cventry{2018 Jan. - 2018 Oct.}
        {Contractor - Leader of Science and Technology}
        {\textsc{Planetario de Bogotá}}
        {Bogotá, Colombia}{}
        {The main objectives of this contract were; to create activities for science divulgation,
          to give conceptual support to the ongoing activities and lead two groops of The Planetarium
          Astronomy Club (8-12 and 13-17 years-old children).}

\cventry{2017 Oct. - 2017 Dec.}
        {Contractor - Developer}
        {\textsc{Guarumo}}{Bogotá, Colombia}{}
        {I worked at Guarumo developing chatbots based on the Facebook Plataform. Thoose chatbots
          where able to appoint dates, sell products and make querys depending on the customer necessities.
          The Facebook Chatbots are developed on Node-JS, the back-end is based on Apache or Nginx}

\cventry{2017 May. - 2017 Aug.}
        {Back-End and Front-End Main Developer}{\textsc{PedroApp Startup}}
        {Bogot\'a, Colombia}{}
        {I have learned Java for Android mobile apps, some basics of Apache2 running on
          Ubuntu Server 14.10 and JavaScript. We have participated in the Apps.CO/MINTIC official contest 
          ``Convocatoria a Equipos de Emprendedores para la Fase de Descubrimiento de
          Negocios TIC de la Iniciativa APPS.CO'' with the project ``Pedro App''.
          The purpose of this contest was to give teaching and accompaniment for new entrepreneur 
          teams.\\
          Ministerio de Tecnolog\'ias de la informaci\'on y las Comunicaciones (MINTIC).
          \href{https://apps.co/comunidad/ver/1785/pedro-e-commerce-multiplataforma-web-app-whatsapp-}
               {\textcolor{blue}{Página del proyecto en Apps.CO.}}
        }
\cventry{2016-II}
        {Adjunct Instructor}
        {\textsc{Universidad Manuela Beltr\'an}}
        {Bogot\'a, Colombia}{}
        {Full time auxiliar instructor. I taught experimental physics courses for first and
          second year students of health sciences and engineering.}
        
\cventry{2015-II}{Graduate Teaching Assistant}{\textsc{Universidad de Los Andes}}
        {Bogot\'a, Colombia}{}
        {Assistant Lecturer of \textbf{Herramientas Computacionales (Computation Tools)} for 
          Physics and  Science. This course requires object-oriented programming knowledge.
          I worked with Juan David Orjuela (Physics Ph.D. student at Uniandes) making some youtube tutorial
          videos in order to improve the time on classroom.
          \\Dedication: 8 hours per week.
          \newline
          \href{https://github.com/ComputoCienciasUniandes/HerramientasComputacionales/tree/master/Lectures/98.Python}
               {\textcolor{blue}{Website, videos, resources and exercises.}}
               \newline{}
               Syllabus:
               \begin{itemize}
               \item UNIX systems basics.
               \item Introducing the Python programming language.
               \item Algorithms
               \item Numerical Methods (derivation, integration, differential equations)
               \item Solving Physical Systems
        \end{itemize}}

\cventry{2015-II}
        {Graduate Teaching Assistant}
        {\textsc{Universidad de Los Andes}}{Bogot\'a, Colombia}{}
        {Graduate Teaching Assistant of \textbf{M\'etodos Computacionales (Computation Methods)} 
          for Physics and Science using Python and C. In this course Advanced numerical methods 
          are taugth, like Runge-Kutta in coupled differential equations, finite differences and
          Monte Carlo applications.\\
          Dedication: 4 hours per week.}
        
%------------------------------------------------

\cventry{2013-II 2014-II}
        {Graduate Teaching Assistant}
        {\textsc{Universidad de Los Andes}}
        {Bogot\'a, Colombia}{}
        {Graduate Teaching Assistant of \textbf{Experimental Physics I \& II} and 
          \textbf{Physics II (Complementary Section)} for Science and Engineering.}

%------------------------------------------------

\cventry{2013-I}
        {Teacher}
        {\textsc{Batakl\'an Corporaci\'on de Artes}}
        {Bogot\'a, Colombia}{}
        {Teacher of \textbf{Art and Science} for elementary and high school.}

\cventry{2013-I}{Teacher}
        {\textsc{Fundaci\'on San Jos\'e}}
        {Bogot\'a, Colombia}{}
        {Lecturer of \textbf{Mathematics and Statistics} for Business Management and Engineering.}

%------------------------------------------------

\cventry{2009--2010}
{Astronomy Auxiliar Undergrad-Student}
{\textsc{Universidad Nacional de Colombia}}
{Bogot\'a, Colombia.}{}
{Course ``Astronom\'ia Para Todos'' (Astronomy for Everybody) at Observatorio 
Astron\'omico Nacional.}

%----------------------------------------------------------------------------------------
%	COMPUTER SKILLS SECTION
%----------------------------------------------------------------------------------------

\section{Computer skills}

Please check some projects in my repo at \url{https://github.com/flgomezc}
\medskip

\begin{itemize}
\item Python, Ipython Notebook, \LaTeX. This is the repo of my current research for
  the M.Sc. Thesis: \href{https://github.com/flgomezc/master_thesis}{flgomezc/master\_thesis},
  a new algorithm using the $\beta$-skeleton graph applied to cosmology in the study of V
  oids in the Large Scale Structure.

\item Facebook Bot (Node-JavaScript) \href{https://github.com/flgomezc/pedrobot}
  {flgomezc/pedrobot}. This is part of an startup that used the Facebook Messenger
  plataform to make online sales.
  
\item Java (Android) \href{https://github.com/flgomezc/movies-app}{flgomezc/movies-app}
  is a small app that connects to the open movie database \href{https://www.themoviedb.org/}
  {themoviedb.org/} to retrieve info from the latests movies.

\end{itemize}

\medskip

\cvitem{Basic}{R, PHP, Apache2}
\cvitem{Intermediate}{\textsc{C++}, Java, MySQL, GIT and Unix-like OS.}
\cvitem{Advanced}{\textsc{Python} with the IPython Notebook (Jupyter), Matplotlib and 
Numpy modules. \LaTeX, \textsc{JavaScript} and Asynchronous programming with the
Facebook API for Messenger Chatbots}


%----------------------------------------------------------------------------------------
%	EDUCATION SECTION
%----------------------------------------------------------------------------------------

\section{Education}


\cventry{(2013-2014) (2018-Up to Date)}
{M.Sc. Physics (Ongoing)}
{Universidad de Los Andes}
{Bogot\'a, Colombia}{}{I started here my formation in Computational Astrophysics. After
  one year I decided to move to the PhD. program, postponing the Master's degree.
  In 2018 I returned tor the M.Sc. program. I am working currently on Large Scale
  Structure of the Universe, developing an algortythm to identify Cosmic Voids in
  dark matter halo catalogs from simulations
  (\href{https://lgarrison.github.io/AbacusCosmos/}{Abacus}) and low redshift
  surveys (\href{https://classic.sdss.org/dr7/}{SDSS})}

\cventry{2014--2015}
{PhD. Physics (Retired)}
{Universidad de Los Andes}
{Bogot\'a, Colombia}{}{
My research field is Computational Astrophysics, dark matter and galaxy formation in 
early universe. I was internship student at the Purdue University (Indiana, USA) by one 
semester (2015-I).
I attended the Dark Energy Spectroscopic Instrument (DESI) International Collaboration May 
2015 Meeting at the FERMILAB (Illinois, USA). 
Also I have assisted to the workshop nIFTy Cosmology: Numerical Simulations for Large 
Surveys hosted by Universidad Aut\'onoma de Madrid, (Madrid, Espa\~na, 2014-summer). 
I was retired from the PhD. program due to health issues.}

\cventry{2013}{B.Sc. Physics}{Universidad Nacional de Colombia}{Bogot\'a, Colombia}{}{
As undergrad student I have explored computational and experimental physics areas. My
degree thesis was on experimental physics, measuring ion beam currents in a low energy
particle accelerator.
}

\bigskip

\subsection{Masters Thesis (In Progress)}
\cvitem{Title}{\emph{A Large Scale Structure Void Identifier for Galaxy Surveys
  Based on the $\beta$-Skeleton Graph Method}.
}
\cvitem{Adviser}{Professor Jaime Forero-Romero}
\cvitem{Description}{
  We have developed a new algorithm based on the $\beta$-Skeleton graph to find voids in
  the LSS. The Beta-Skeleton has been widely used on machine learning, optimization
  algorithms and image recognition and processing. It has been introduced recently in
  the LSS analysis as a fast tool to identify LSS filamentary structure, and now can
  detect voids also. After identify the voids in the catalgos, we will make some
  predictions for the upcoming DESI void population, using the voids morphology
  statistics as cosmological test.}
\cvitem{Keywords}{Large Scale Structure, Computational Astrophysics.}

\medskip

\subsection{Bachelor Thesis}
\cvitem{Title}{Mass Characterization of the $H_3 ^+$ and $H ^-$ Ion Beam From a Plasmatron
Ion Source. 2012}
\cvitem{Director}{Professor Gustavo Mart\'inez Tamayo}
\cvitem{Description}{This work studied how the change of the geometry of the thermionic 
emission filament in the UNAC-Plasmatron (ion source) generates a change in the ion 
production and the ion beam.}
\cvitem{Keywords}{Plasma, Ion Beam, $H ^-$ Ion, Experimental Physics.}

\medskip

\subsection{Other}

\cvitem{January 2016}{\textbf{PADI Scuba Diver}, Reef Shepherd Professional Dive Center, 
Santa Marta, Colombia. }

\cvitem{2015-I}{Internship at \textbf{Purdue University, Indiana, USA}.  Colciencias 
scholarship holder.}

\cvitem{July 2014}{\textbf{nIFTy Cosmology: Numerical Simulations for Large Surveys} 
Universidad Aut\'onoma de Madrid, Madrid, Espa\~na.
\newline
\url{http://popia.ft.uam.es/nIFTyCosmology/Home.html}}

\cvitem{January 2014}{\textbf{Cosmology on the Beach: Essential Cosmology for the 
Next Generation.} Berkeley Center for Cosmological Physics and Advanced Institute for 
Cosmology (M\'exico). Cabo San Lucas, M\'exico.
\newline
\url{http://bccp.berkeley.edu/beach_program/index2014.html}
}

\cvitem{June 2013}{\textbf{Workshop Astronom\'ia en Los Andes}, Uniandes, Bogot\'a, 
Colombia}

\cvitem{2012}{\textbf{Circuits and Electronics 6.002x.} Online course of the MIT.
\newline
\url{http://6002x.mitx.mit.edu/info}}

\cvitem{October 2010}{\textbf{Workshop on Physics and Technology at CERN}, UNAL, 
Bogot\'a, Colombia}
\cvitem{August-2010}{\textbf{Escuela de Astronom\'ia Extragal\'actica} Observatorio 
Astron\'omico Nacional, Bogot\'a , Colombia.}

\cvitem{October 2009}{\textbf{XXIII Congreso Nacional de F\'isica, Universidad del 
Magdalena}, Santa Marta, Colombia.
\newline
\url{www.sociedadcolombianadefisica.org.co/pag/eventos2009.php}}

\cvitem{2008}{\textbf{Short Tale Workshop``Ciudad de Bogot\'a 2008.''} Fellow of the
Capital District in the workshop adjoint to RENATA (Red Nacional de Talleres de 
Escritura Creativa)}


%----------------------------------------------------------------------------------------
%	AWARDS SECTION
%----------------------------------------------------------------------------------------
\section{Awards}
\cvitem{2011-II}{Mejor Saber-Pro 2011-2 en F\'isica. ICFES.
\newline
\href{http://www.icfes.gov.co/index.php/instituciones-educativas/saber-pro/mejores-saber-pro}
{Top National Physics Bachelor Student. The list is available at www.icfes.gov.co}
}

%------------------------------------------------

\section{Publications}
\cvitem{2013}{\textbf{Experiments with Polygonal and Polyhedral Resistive Structures.}
\textit{R. Beltr\'an, F. G\'omez, R Franco, J-Alexis Rodr\'iguez and F. Fajardo,}
Latin American Journal of Physics Education Vol 7, Issue 4, Dec 2013, 621-624
\newline
\url{www.lajpe.org/index_dec2013.html}
}

\cvitem{2010}{\textbf{Cenizas en el And\'en. Antolog\'ia de Cuento Urbano.} 
City Short Tales Antology, story tittle: \textbf{Imag\'inate} as Filipo Rviz. 
ISBN 978-958-44-4585-8. 
\newline
\url{https://es.wikipedia.org/wiki/Cenizas\_en\_el\_and\%C3\%A9n}}


%----------------------------------------------------------------------------------------
%	LANGUAGES SECTION
%----------------------------------------------------------------------------------------
\section{Languages}
\cvitemwithcomment{Spanish}{Native Speaker}{}
\cvitemwithcomment{English}{B2 Level}
{Common European Framework of Reference. BULATS Test. 2011.}

%----------------------------------------------------------------------------------------
%	INTERESTS SECTION
%----------------------------------------------------------------------------------------

\section{Interests}

\renewcommand{\listitemsymbol}{-~} % Changes the symbol used for lists

\cvlistdoubleitem{Reading}{Writing}
\cvlistdoubleitem{Swimming}{Diving}
\cvlistdoubleitem{Guitar}{Riding Bike}
%\cvlistitem{Running}

%----------------------------------------------------------------------------------------

\end{document}
