%%%%%%%%%%%%%%%%%%%%%%%%%%%%%%%%%%%%%%%%%
% "ModernCV" CV and Cover Letter
% LaTeX Template
% Version 1.2 (25/3/16)
%
% This template has been downloaded from:
% http://www.LaTeXTemplates.com
%
% Original author:
% Xavier Danaux (xdanaux@gmail.com) with modifications by:
% Vel (vel@latextemplates.com)
%
% License:
% CC BY-NC-SA 3.0 (http://creativecommons.org/licenses/by-nc-sa/3.0/)
%
% Important note:
% This template requires the moderncv.cls and .sty files to be in the same 
% directory as this .tex file. These files provide the resume style and themes 
% used for structuring the document.
%
%%%%%%%%%%%%%%%%%%%%%%%%%%%%%%%%%%%%%%%%%

%----------------------------------------------------------------------------------------
%	PACKAGES AND OTHER DOCUMENT CONFIGURATIONS
%----------------------------------------------------------------------------------------

\documentclass[12pt,letterpaper,sans]{moderncv} 
% Font sizes: 10, 11, or 12; paper sizes: a4paper, letterpaper, a5paper, legalpaper, 
% executivepaper or landscape; font families: sans or roman

\usepackage[utf8]{inputenc}

\moderncvstyle{casual} 
% CV theme - options include: 'casual' (default), 'classic', 
% 'oldstyle' and 'banking'
\moderncvcolor{blue} 
% CV color - options include: 'blue' (default), 'orange', 'green', 'red', 'purple', 
% 'grey' and 'black'

\usepackage{lipsum} % Used for inserting dummy 'Lorem ipsum' text into the template

\usepackage[scale=0.75]{geometry} % Reduce document margins
%\setlength{\hintscolumnwidth}{3cm} % Uncomment to change the width of the dates column
%\setlength{\makecvtitlenamewidth}{10cm} % For the 'classic' style, uncomment to adjust 
% the width of the space allocated to your name

\setlength\parindent{0pt}
\newcommand{\forceindent}{\leavevmode{\parindent=1em\indent}} 
% Filis Var: To indent a single paragraph

%\usepackage[]{hyperref}
%----------------------------------------------------------------------------------------
%	NAME AND CONTACT INFORMATION SECTION
%----------------------------------------------------------------------------------------

\firstname{Felipe Leonardo}
\familyname{Gómez Cortés}

% All information in this block is optional, comment out any lines you don't need
\title{Curriculum Vitae}
\address{Calle 23c 70-50 Int 19 Apt 404}{Bogotá, Colombia.}
\mobile{(+57) 316 845 5161}
\phone{(+571) 759 5221}
%\fax{(000) 111 1113}
\email{fl.gomez10@uniandes.edu.co}
\homepage{github.com/flgomezc}{https://github.com/flgomezc} 
%\extrainfo{additional information}
%\photo[70pt][0.4pt]{pictures/picture} % The first bracket is the picture height, 
% the second is the thickness of the frame around the picture (0pt for no frame)
%\quote{"A witty and playful quotation" - John Smith}

%----------------------------------------------------------------------------------------

\begin{document}

%----------------------------------------------------------------------------------------
%	COVER LETTER
%----------------------------------------------------------------------------------------
%
% To remove the cover letter, comment out this entire block
%
%\clearpage
%
%\recipient{HR Department}{Corporation\\123 Pleasant Lane\\12345 City, State} 
            % Letter recipient
%\date{\today} % Letter date
%\opening{Dear Sir or Madam,} % Opening greeting
%\closing{Sincerely yours,} % Closing phrase
%\enclosure[Attached]{curriculum vit\ae{}} % List of enclosed documents
%
%\makelettertitle % Print letter title

%\lipsum[1-3] % Dummy text

%\makeletterclosing % Print letter signature

%\newpage

%----------------------------------------------------------------------------------------
%	CURRICULUM VITAE
%----------------------------------------------------------------------------------------

\makecvtitle % Print the CV title


%----------------------------------------------------------------------------------------
%	PROFILE SECTION
%----------------------------------------------------------------------------------------

\section{Perfil}

Soy físico, estudio actualmente Maestría en Ciencias-Física en la Universidad de Los Andes.
Mi área de investigación es Astrofísica Computacional. He desarrollado fuertes
habilidades para resolver problemas físicos mediante el modelamiento y
análisis de datos usando herramientas computacionales programando en Python en sistemas
operativos tipo UNIX.
\par
\forceindent He trabajado como docente a nivel universitario, como Asistente Graduado
y como Profesor Asistente. También he trabajado en divulgación científica como contratista
en el Planetario de Bogotá.

\forceindent
Como asistente graduado dicté en la Universidad de Los Andes en 2015 dos cursos de
programación orientados para estudiantes de ciencias e ingeniería: dos secciones del curso
``Herramientas Computacionales'' y profesor asistente en la sección complementaria
del curso ``Métodos Computacionales''. En años anteriores dicté clases de física experimental
I y II, y secciones complementarias de Física II (electricidad, magnetismo y termodinámica).

\forceindent
He trabajado también como desarrollador de software, tanto en back-end como en front-end.
Participé en proyectos donde aprendí Apache, programación de aplicaciones en Java para
Android, bases de datos en MySQL, PHP y NodeJS para desarrollar chatbots conectados a la
plataforma de Messenger.
\par

%----------------------------------------------------------------------------------------
%	WORK EXPERIENCE SECTION
%----------------------------------------------------------------------------------------

\section{Experiencia Laboral}

\cventry{2018 Ene. - 2018 Oct.}
        {Contratista - Lider del área de Ciencia y Tecnología}
        {\textsc{Planetario de Bogotá}}
        {Bogotá, Colombia}{}
        { Uno de los objetivos principales de este contrato era apoyar al IDARTES en el
          desarrollo de actividades de divulgación científica para el público, fortalecer
          procesos de mediación brindando apoyo conceptual, y dirigir dos grupos (de 8 a
          12 aÑOS y de 13 a 17 años) del Club Juvenil de Tecnología y Astronomía del
          Planetario de Bogotá. Entre estas experiencias desarrollé un curso de
          programación en Python relámpago de 16 horas para jóvenes de 13 a 17 años,
          que incluía visualización de datos astronómicos con librerías de Matplotlib.
          \url{https://github.com/flgomezc/planetario_bogota}
          \\
          Supervisor: Edgar Cipagauta, profesional especializado. Tel: 3795750 ext. 2504.
        }


\cventry{2017 Oct. \\ 2017 Dic.}
        {Contratista - Desarrollador de Software}
        {\textsc{Guarumo}}{Bogotá, Colombia}{}
        { Trabajé en Guarumo desarrollando principalmente chatbots para la plataforma Facebook. Entre las
          distintas funciones, los chatbots podían realizar búsquedas en bases de datos,
          agendar y modificar citas. Parte del trabajo se realizó en servidores Apache incluyento
          PHP y bases de datos MySQL. Los bots se desarrollaron en JavaScript, con
          intregración a servicios de inteligencia artificial como Watson de IBM. \\
          Otro proyecto en el cuál trabajé
          involucraba la visualización de gran volúmen de datos de transporte individual de pasajeros,
          utilizando librerías de Python.\\
          Brindé también mis servicios de investigación académica para estudiar el impacto de las encuestas
          por medios electrónicos.\\
          Jefe inmediato: Julián Villarreal, gerente de producto. Tel: 4813669, 318 7827666. E-mail:
          \url{producto@guarumo.com}}

          
\cventry{2017 May.\\2017 Ago.}
        {Desarrollador de producto}{\textsc{PedroApp Startup}}
        {Bogot\'a, Colombia}{}
        {Participé en la ``Convocatoria a Equipos de Emprendedores para la Fase de Descubrimiento de
          Negocios TIC de la Iniciativa APPS.CO'' con el proyecto ``Pedro App'' donde adquirí conocimientos
          sobre la metodología Lean Start Up y apliqué mis conocimientos en programación para el desarrollo
          del proyecto.\\
          Ministerio de Tecnolog\'ias de la informaci\'on y las Comunicaciones (MINTIC).
          \href{https://apps.co/comunidad/ver/1785/pedro-e-commerce-multiplataforma-web-app-whatsapp-}
               {\textcolor{blue}{Página del proyecto en Apps.CO.}}
               \\
               Mentor:  Julián Villarreal. Tel: 4813669, 318 7827666. E-mail:
                         \url{producto@guarumo.com}
        }

\cventry{2016-II}
        {Auxiliar de Docencia}
        {\textsc{Universidad Manuela Beltr\'an}}
        {Bogot\'a, Colombia}{}
        { Trabajé tiempo completo dictando 13 cursos de física experimental para estudiantes de
          primer y segundo año de ciencias de la salud e ingeniería.}
        
\cventry{2015-II}
        {Asistente Graduado}{\textsc{Universidad de Los Andes}}
        {Bogot\'a, Colombia}{}
        { Dicté dos sesiones del curso \textbf{Herramientas Computacionales} en Python para ciencias e ingeniería.
          Junto con Juan David Orjuela (Estudiante del Doctorado en Física de UNIANDES) trabajamos con el
          material ya construido por profesores anteriores y basado en este trabajo, desarrollamos
          una serie de vídeos que montamos en la plataforma de youtube para cambiar la metodología de
          clase, permitiendo a los estudiantes revisar los temas antes de la sesión y así aprovechar
          el tiempo de clase como sesiones prácticas donde los estudiantes resuelven los ejercicios de
          programación y los profesores guían este proceso y resuelven dudas puntuales sobre el código.
          \\Dedicación: 8 horas por semana.
          \newline
          \href{https://github.com/ComputoCienciasUniandes/HerramientasComputacionales/tree/master/Lectures/98.Python}
               {\textcolor{blue}{Repositorio en Github, URL de videos, material de apoyo y ejercicios.}}
               \newline{}
               Syllabus:
               \begin{itemize}
               \item Introducción a sistemas operativos tipo UNIX.
               \item Introducción al lenguaje de programación Python 2.0.
               \item Algoritmos
               \item Visualización de datos (Matplotlib)
               \item Métodos Numéricos (derivación, integración, solución de ecuaciones diferenciales)
               \item Solución de problemas físicos.
               \end{itemize}
        Ese mismo semestre fui monitor en el curso de últimos semestres de pregrado
        \textbf{Métodos Computacionales} para estudiantes de física y geociencias, apoyando las
        sesiones complementarias del curso y con tiempo en oficina para resolver dudas individuales
        de los estudiantes en el desarrollo de ejercicios propuestos en clase. En este curso se
        trabajó con métodos numéricos avanzados, como solución de ecuaciones diferenciales acopladas
        mediante el método Runge-Jutta, método de diferencias finitas para resolver ecuaciones
        diferenciales de varias variables, análisis de componentes principales (PCA) y aplicaciones
        del método Monte-Carlo.
        Dedicación: 4 horas por semana.
        \\
        Profesor principal: Jaime Forero, Departamento de Física, Uniandes.
        }

%------------------------------------------------

\cventry{2013 Jul\\ 2014 Nov}
        {Asistente Graduado}
        {\textsc{Universidad de Los Andes}}
        {Bogot\'a, Colombia}{}
        {Profesor de \textbf{Física Experimental I \& II} (mecánica, electricidad, magnetismo
          y termodinámica) para ciencias e ingeniería. Profesor de la sección complementaria
          de \textbf{Física II (Complementary Section)} para ciencias, ingeniería y medicina.}

%------------------------------------------------

\cventry{2013-I}
        {Profesor en Bachillerato}
        {\textsc{Batakl\'an Corporaci\'on de Artes}}
        {Bogot\'a, Colombia}{}
        {Profesor del curso \textbf{Arte y Ciencia} para estudiantes de bachillerato en el colegio
          distrital Manuel del Socorro Rodríguez en el programa de jornada complementaria \textbf{``40x40''}
          Este curso tuvo un énfasis en Física y Música, aprovechando mi formación en física y que el grupo de
          estudiantes hacían parte de la banda de Rock y la Banda de Guerra del colegio, también parte del
          programa de jornada complementaria.
          Vimos conceptos como el péndulo y los
          estudios que hizo Galileo, el funcionamiento del oído humanano, el rango auditivo, la escala
          pitagórica musical, resonancia en la vida diaria (puente Takoma Narrow) y en los instrumentos
          musicales.
          \\ Dedicación: 4 horas por semana.}

\cventry{2013-I}{Profesor Universitario}
        {\textsc{Fundaci\'on San Jos\'e}}
        {Bogot\'a, Colombia}{}
        {Profesor de los cursos Matemática Financiera, Estadística Descriptiva y Física (Electromagnetismo)
          para estudiantes de Administración de Empresas e Ingeniería.}

%------------------------------------------------

\cventry{2009 Agosto \\ 2011 Junio}
        {Estudiante Auxiliar de Pregrado}
        {\textsc{Universidad Nacional de Colombia}}
        {Bogot\'a, Colombia}{}
        {Fui monitor del curso de contexto ``Astronomía Para Todos''. Este curso lo ofrece el Observatorio
          Astronómico Nacional para estudiantes de toda la universidad. El curso trata sobre la esfera
        celeste, coordenadas ecuatoriales, evolución estelar, galaxias y la astronomía de hoy en día.}

%----------------------------------------------------------------------------------------
%	COMPUTER SKILLS SECTION
%----------------------------------------------------------------------------------------

\section{Habilidades en Programación}

Tengo un repositorio en Github con algunos de mis proyectos en \url{https://github.com/flgomezc}
\medskip

\begin{itemize}
\item Java (Android) \url{https://github.com/flgomezc/movies-app}
\item Facebook Bot (Node-JavaScript) \url{https://github.com/flgomezc/pedrobot}
\item Python, Ipython Notebook, \LaTeX \url{https://github.com/flgomezc/sfr-dmhm}
\end{itemize}

\medskip

\cvitem{Nivel básico}{R, PHP, Apache2}
\cvitem{Intermedio}{\textsc{C++}, Java, MySQL, GIT and Unix-like OS.}
\cvitem{Avanzado}{\textsc{Python} con IPython Notebook (Jupyter), módulos Matplotlib y 
Numpy. \LaTeX, \textsc{JavaScript} and programación asíncrona}


%----------------------------------------------------------------------------------------
%	EDUCATION SECTION
%----------------------------------------------------------------------------------------

\section{Education}


\cventry{2013--Actualmente}
        {Maestría en Ciencias-Física}
        {Universidad de Los Andes}
        {Bogot\'a, Colombia}{}
        {Mi área de investigación es Astrofísica Computacional, trabajando en con la
          estructura de gran escala del universo a partir de observaciones y simulaciones.
          La propuesta de proyecto de tesis que desarrollaré en 2019 se titula:
          ``A Large Scale Structure Void Identifier for Galaxy Surveys Based on the
          $\beta$-Skeleton Graph Method''. Este documento se encuentra disponible en el
          repositorio: \url{https://github.com/flgomezc/master_thesis}
        }

\cventry{2014--2015}
        {Doctorado en Ciencias-Física (Retirado)}
        {Universidad de Los Andes}
        {Bogot\'a, Colombia}{}{
          El proyecto de investigación en el que trabajé buscaba encontrar una relación
          entre halos de materia oscura y formación de estrellas en galaxias de universo
          temprano. Participé en una pasantía en la Universidad de Purdue (Indiana, EE.UU.)
          con el apoyo de Uniandes y COLCIENCIAS, la reunión ``Dark Energy Spectroscopic
          Instrument (DESI) International Collaboration May 2015 Meeting'' en el FERMILAB
          (Illinois, USA). Asistí al ``Workshop nIFTy Cosmology: Numerical Simulations for
          Large Surveys'' en la Universidad Aut\'onoma de Madrid, (Madrid, Espa\~na,
          2014). Fui retirado del programa de doctorado por problemas de salud.}

\cventry{2003-2012}
        {Físico}
        {Universidad Nacional de Colombia}
        {Bogot\'a, Colombia}{}
        { Como estudiante de pregrado desarrolé mi interes por la física computacional
          y la física experimental. Mi trabajo de grado se desarrolló en física
          experimental, midiendo corrientes iónicas en un acelerador de partículas de baja
          energía.}

\bigskip

\subsection{Proyecto de tesis de Doctorado}
\cvitem{Título}{\emph{A Connection Between Star Formation Rate and Dark Matter Halos at 
z$\sim$6}.}
\cvitem{Director}{Profesor Jaime Forero-Romero, Uniandes.}
\cvitem{Descripción}{Este trabajo relaciona materia oscura y formación estelar en
  galaxias de universo temprano. Se usan catálogos de galaxias de observaciones del telescopio
  espacial Hubble y otros telescopios en tierra, junto con catálogos de halos de materia
  oscura de la simulación cosmológica de gran escala Bolshoi, para relacionarlos según
  el método de cadenas de Markov Monte-Carlo.
  Repo: \url{https://github.com/flgomezc/sfr-dmhm}}
\cvitem{Keywords}{Large Scale Structure, Galaxy Evolution, Computational Astrophysics.}

\medskip

\subsection{Tesis de pregrado en Física}
\cvitem{Título}{Mass Characterization of the $H_3 ^+$ and $H ^-$ Ion Beam From a Plasmatron
Ion Source. 2012}
\cvitem{Director}{Profesor Gustavo Mart\'inez Tamayo, UNAL.}
\cvitem{Descripción}{Este grabajo estudió cómo el cambio de la geometría del filamento
  de emisión termoiónica en el la fuente de iones UNAC-Plasmatrón generaba un cambio
  en la producción iónica de la fuente y el haz de partículas del acelerador de partículas.}
\cvitem{Keywords}{Plasma, Ion Beam, $H ^-$ Ion, Experimental Physics.}

\medskip

\subsection{Otros}

\cvitem{2016-Enero}{\textbf{PADI Scuba Diver}, Reef Shepherd Professional Dive Center, 
Santa Marta, Colombia. }

\cvitem{2015-I}{Pasantía en \textbf{Purdue University, Indiana, USA}. Con el apoyo de
  Uniandes y Colciencias.}

\cvitem{2014-Julio}{\textbf{nIFTy Cosmology: Numerical Simulations for Large Surveys} 
Universidad Aut\'onoma de Madrid, Madrid, Espa\~na.
\newline
\url{http://popia.ft.uam.es/nIFTyCosmology/Home.html}}

\cvitem{2014-Enero}{\textbf{Cosmology on the Beach: Essential Cosmology for the 
Next Generation.} Berkeley Center for Cosmological Physics and Advanced Institute for 
Cosmology (M\'exico). Cabo San Lucas, M\'exico.
\newline
\url{http://bccp.berkeley.edu/beach_program/index2014.html}
}

\cvitem{2013-Junio}{\textbf{Workshop Astronom\'ia en Los Andes}, Uniandes, Bogot\'a, 
Colombia}

\cvitem{2012}{\textbf{Circuits and Electronics 6.002x.} Curso en línea del MIT.
\newline
\url{http://6002x.mitx.mit.edu/info}}

\cvitem{2010 Octubre}{\textbf{Workshop on Physics and Technology at CERN}, UNAL, 
Bogot\'a, Colombia}
\cvitem{2010-Agosto}{\textbf{Escuela de Astronom\'ia Extragal\'actica} Observatorio 
Astron\'omico Nacional, Bogot\'a , Colombia.}

\cvitem{2009-Octubre}{\textbf{XXIII Congreso Nacional de F\'isica, Universidad del 
Magdalena}, Santa Marta, Colombia.
\newline
\url{www.sociedadcolombianadefisica.org.co/pag/eventos2009.php}}

\cvitem{2008}{\textbf{Taller de cuento ``Ciudad de Bogot\'a 2008.''} Becario del distrito
  en el taller de RENATA (Red Nacional de Talleres de 
Escritura Creativa)}


%----------------------------------------------------------------------------------------
%	AWARDS SECTION
%----------------------------------------------------------------------------------------
\section{Premios}
\cvitem{2011-II}{Mejor Saber-Pro 2011-2 en F\'isica. ICFES.
\newline
\href{http://www.icfes.gov.co/index.php/instituciones-educativas/saber-pro/mejores-saber-pro}
{Mejores estudiantes de pregrado. Listado disponible en \ref{www.icfes.gov.co}}
}

%------------------------------------------------

\section{Publicaciones}
\cvitem{2013}{\textbf{Experiments with Polygonal and Polyhedral Resistive Structures.}
\textit{R. Beltr\'an, F. G\'omez, R Franco, J-Alexis Rodr\'iguez and F. Fajardo,}
Latin American Journal of Physics Education Vol 7, Issue 4, Dec 2013, 621-624
\newline
\url{www.lajpe.org/index_dec2013.html}
}

\cvitem{2010}{\textbf{Cenizas en el And\'en. Antolog\'ia de Cuento Urbano.} 
cuento: \textbf{Imag\'inate} firmado como Filipo Rviz. 
ISBN 978-958-44-4585-8. 
\newline
\url{https://es.wikipedia.org/wiki/Cenizas\_en\_el\_and\%C3\%A9n}}


%----------------------------------------------------------------------------------------
%	LANGUAGES SECTION
%----------------------------------------------------------------------------------------
\section{Lenguajes}
\cvitemwithcomment{Español}{Lengua Nativa}{}
\cvitemwithcomment{Inglés}{B2 Level}{Common European Framework of Reference. BULATS Test. 2011.}

%----------------------------------------------------------------------------------------
%	INTERESTS SECTION
%----------------------------------------------------------------------------------------

\section{Intereses}

\renewcommand{\listitemsymbol}{-~} % Changes the symbol used for lists

\cvlistdoubleitem{Lectura}{Escritura}
\cvlistdoubleitem{Natación}{Buceo}
\cvlistdoubleitem{Guitarra}{Ciclismo}
%\cvlistitem{Running}

%----------------------------------------------------------------------------------------

\end{document}
